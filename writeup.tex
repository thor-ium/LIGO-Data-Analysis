%% Full length research paper template
%% Created by Simon Hengchen and Nilo Pedrazzini for the Journal of Open Humanities Data (https://openhumanitiesdata.metajnl.com)

\documentclass{article}
\usepackage[english]{babel}
\usepackage[utf8]{inputenc}
% \usepackage{johd}

\title{ASTR3800 Final Project Writeup}

\author{Thor Breece}

\date{12/8/2022} %leave blank

\begin{document}

\maketitle

\begin{abstract} 
\noindent  
Through a combination of data science, statistical analysis, machine learning and intuition, I analyzed data collected
by LIGO. Throughout the project, I analyzed the Event Demographics of the LIGO data as a whole, the neutron star - neutron star merger GW170817, and discovered 
hidden correlations between variables within the data set. This helped me to understand the limitations of LIGO, how standard sirens, often sources of LIGO GW signals, 
are used to calculate the hubble constant, and the correlation between the Signal to Noise ratio and Luminosity Distance.
\end{abstract}



\section{Gravitational Waves and LIGO}


Describe the context and motivation of your paper.

\subsection{In-text citations}
This journal uses a style based on the APA system (see \href{https://openhumanitiesdata.metajnl.com/about/submissions/#References}{here}). \\
The following are some basic citation commands in \LaTeX: \\

\noindent
\verb|\citet| $\rightarrow$ \citet{jenset&mcgil}\\
\verb|\citet| $\rightarrow$ \citet{australiashealth}\\
\verb|\citet| $\rightarrow$ \citet{shree-a}\\
\verb|\citep| $\rightarrow$ \citep{fabricius-hansen2012b}\\
\verb|\citealp| $\rightarrow$ (\citealp{eckhoff2018a})\\
\verb|\citealp| $\rightarrow$ (\citealp{eckhoff2018a}; \citealp{fabricius-hansen2012b}; \citealp{shree-a})\\

\subsubsection{Other simple functions}
To add bullet points:

\begin{itemize}
    \item Some point
    \item Another point
\end{itemize}

\noindent Or numbered points:

\begin{itemize}
    \item[1.] Some numbered point
    \item[2.] Another numbered point
\end{itemize}

\noindent This is an example of footnote\footnote{This is a footnote}. \\

\noindent This is a simple table:

\begin{table}[H]
\centering % Label your table accordingly
\caption{\label{tab1} A caption.}
\begin{tabular}{cccc}
\hline
1 & 2 & 3 & 4 \\
\hline
a & b & c & d\\
e & f & g & h\\
\hline
\end{tabular}
\end{table}

\noindent Please refer to your table using: Table \ref{tab1}.\\

\noindent To add a figure, upload the figure into the \texttt{images} folder, and then embed it:

\begin{figure}[H]
\centering
\includegraphics{images/image.jpeg}
\caption{\label{fig1}JOHD's logo.}
\end{figure}

\noindent To resize the figure:

\begin{figure}[H]
\centering
\includegraphics[width=0.2\textwidth]{images/image.jpeg}
\caption{\label{fig2}JOHD's logo.}
\end{figure}

\begin{figure}[H]
\centering
\includegraphics[width=0.8\textwidth]{images/image.jpeg}
\caption{\label{fig3}JOHD's logo.}
\end{figure}

\noindent Please refer to your figures as: Figure \ref{fig1}, Figure \ref{fig2}, etc.


\section{Dataset description}
Here you can provide, if applicable, information about the dataset(s) whose creation, collection, management, access, processing or analysis have been discussed in this paper, following this schema:
\paragraph{Object name} Typically the name of the file or file set in the repository.
\paragraph{Format names and versions} E.g., ASCII, CSV, Autocad, EPS, JPEG, Excel, SQL, etc.
\paragraph{Creation dates} The start and end dates of when the data was created (YYYY-MM-DD).
\paragraph{Dataset creators} Please list anyone who helped to create the dataset (who may or may not be an author of the data paper), including their roles (using and affiliations).
\paragraph{Language} Languages used in the dataset (i.e., for variable names etc.).
\paragraph{License} The open license under which the data has been deposited (e.g., CC0). 
\paragraph{Repository name} The name of the repository to which the data is uploaded. E.g., Figshare, Dataverse, etc. 
\paragraph{Publication date} If already known, the date in which the dataset was published in the repository (YYYY-MM-DD).

\section{Method}
Describe the methods used in the study.

\section{Results and discussion}
Describe and discuss the results of the study.

\section{Implications/Applications}
Provide information about the implications of this research and/or how it can be applied.

\section*{Acknowledgements}
Please add any relevant acknowledgements to anyone else that assisted with the project in which the data was created but did not work directly on the data itself.


\bibliographystyle{johd}
\bibliography{bib}

\end{document}